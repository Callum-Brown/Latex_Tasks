\documentclass[11pt]{article}

\usepackage[nonewpage, afterindex]{indextools}
\usepackage{hyperref}
\usepackage{lipsum}

\makeindex[name=ch1]
\makeindex[name=ch2]
\makeindex[name=ch3, title=Pages of Reference]

%\renewcommand{\indexname}{\sectioname\ thesection\ Index}

%\preto\endtheindex{\par\nobreak\noindent\hfill$\blacksquare$}

\newcommand{\Refindex}[1]{%
	Chapter
	\ref{Chp#1}
	\index[ch#1]{Chapter #1}
}%

\newcommand{\PrintnLabel}[1]{%
	\printindex[ch#1]
	\label{Chp#1}
}%

\begin{document}

\section{Chapter}
\PrintnLabel{1}
%\printindex[ch1]
%\label{Chp1}
\lipsum[1-8]
\section{Chapter}
\PrintnLabel{2}
%\printindex[ch2]
%\label{Chp2}
\lipsum[9-12]
\section{Chapter}
\PrintnLabel{3}
%printindex[ch3]
%\label{Chp3}
\lipsum[13-21]



Here we mention the name of a famous mathematician, Carl Friedrich 
Gauss  \Refindex{1},  \Refindex{1}
who is very well known
for his work  \Refindex{2}
 \Refindex{3}
\end{document}